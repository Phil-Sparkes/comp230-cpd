% Please do not change the document class
\documentclass{scrartcl}

% Please do not change these packages
\usepackage[hidelinks]{hyperref}
\usepackage[none]{hyphenat}
\usepackage{setspace}
\doublespace

% You may add additional packages here
\usepackage{amsmath}

% Please include a clear, concise, and descriptive title
\title{CPD Report}

% Please do not change the subtitle
\subtitle{COMP230 - CPD}

% Please put your student number in the author field
\author{1608305}

\begin{document}

\maketitle

\section{Pacing in Presentations}
My target is to improve pacing during presentations. This year my presentations have not been to a standard I would like them to be at. Currently I talk too fast trying to reach the end of the presentation as fast as possible. Slowing down the rate I talk at will make my words easier to understand and improve the overall quality of my presentations.  
\newline
\newline
I will do this by conducting a presentation in front of my peers every 2 weeks for 8 weeks. I will ask for immediate feedback directly after I have conducted my presentation. I will also ask for a rating out of 10 on three different subjects; Pacing, question answering and ability to get ideas across. 
\newline
\newline
Presentations are relevant to myself and the course since presentations are often used to convey ideas. To be able to present in a clear manner will provide others with a better understanding of my ideas.
\newline
\newline
I will commit 2 hours a week on learning presentation skills by watching videos and self-evaluation as well as an hour making a presentation on a topic I am interested in during that current week.

\section{Prototyping fast with team}
My target is to improve the speed in which I prototype team projects. During the prototyping section of the group project I found myself struggling to create prototypes in such a brief time period, this meant that out of the three games we prototyped I had only worked on one. Speeding up the time I can create a prototype will mean that I can produce more prototypes in the allotted period.
\newline
\newline
The best way to do this is to produce game prototypes. I will create two prototypes a week for 3 weeks. I will self-assess how well the game prototyping has been done giving each a rating out of 10, I will also write a small report on what went right, what went wrong and what could be done better for each prototype. I will commit 4 hours to each prototype I create.
\newline
\newline
Being able to create prototypes fast means that more ideas could be tested before a final idea is settled on. This will in turn lead to a better final product.

\section{Collecting quantitative and qualitative data}
My next target is to improve my methods of collecting quantitative and qualitative data. For an assignment we had to conduct a playtesting of a game interface and collect quantitative and qualitative data from the players. The data I collected didn’t produce any definitive results. I intend on getting better at creating questionnaires and performing interviews to receive better results. Also getting a wide sample group to take part in the interviews/questionnaires.
\newline
\newline
The way I intend to do this is via my current group game project. The game project is now in the state in which it can be play tested and as a group we have decided to try and get some playtests every week. I will help with the creation of the questionnaires and conduct interviews for these play tests. I will carry on doing this till the game is complete. To measure the effectiveness of this data I will converse with my group and ask them for feedback and asking how I could improve the questionnaires/interviews to achieve better data.
\newline
\newline
With better data we will easily be able to see what players think of our game and how we can balance and improve it.

\section{C++ pointers and references}
Another target I would like to achieve is to gain a better understanding of how pointers and references work in C++. During the graphics and simulation assignment I had trouble getting some segments of the code to work, after a while of trying I managed to get it working using pointers and referencing but admittedly I still do not know why it worked. I intend on getting an understanding of how they work so I can use them more frequently in my code.
\newline
\newline
To achieve this, I will follow an online tutorial on the subject. I will go over my code for graphics and simulation and add in the appropriate pointer then ask for a code review from someone experienced in using them. With the feedback I will make any improvements suggested. I will spend a week doing this with multiple code reviews until I am happy that I have a firm understanding of how they work and when to use them.
\newline
\newline
Learning how to effectively use pointers and referencing in C++ will lead to better quality and more efficient code. It will also help me understand code written by others.


\section{Working in studio and keeping up with work}
My last target is to spend more time working the studio which will also help with keeping up with my work. I have been working more at home than at the studio, this makes me harder to contact for my team which may require help with the game project. Distractions at my home environment lead to me being less productive. There are also many distractions in the studio, but I will try and find a quiet corner and use headphones to block the noise.
\newline
\newline
To make sure that I spend more time in the studio I will timetable myself to be in at least 5 hours every day for the rest of the academic year. I shall keep track of my attendance and see if I am managing to commit to the 5 hours a day.
\newline
\newline
Working in studio will mean that I can work better with my team and help with problems they may find.

\bibliographystyle{ieeetran}
\bibliography{references}

\end{document}
